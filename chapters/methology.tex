%
\section{提案手法}\label{ch:methodology}
%
\subsection{モデル概観}
今回提案したモデルは,以下の図\ref{fig:model}の通りである.

提案モデルの目的は,画像と文章で発信された情報に対して,
正しいニュースか・フェイクニュースか・ジョークニュースかを分類するために,
必要な特徴表現を学習することであった.
提案モデルは複合特徴量抽出器とニュース分類器の大きく2部分に分けることができた.
まず複合特徴量抽出器は,今回扱う情報が文章と画像を含むため,
各メディアに対して特徴化する抽出器があった.
その後それぞれの特徴を1つに連結し,複合特徴を形成した.
複合特徴はニュース分類器に送られ,最終的には3カテゴリのどれに該当するかが判断された.
% 
\begin{figure}[H]
    \centering
    \includegraphics[width=\linewidth]{images/methodology.pdf}
    \caption{提案モデル図.青: 文章特徴量抽出器,橙: 画像特徴抽出器,紫: ニュース分類器.}
    \label{fig:model}
\end{figure}
%
