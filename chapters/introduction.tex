% 背景[フェイクニュースの自動判別の必要性+ジョークニュースを区別する必要性]+想定環境の説明(導入との被りに注意)
\section{はじめに}\label{ch:introduction}
%
% フェイクニュース
昨今のSNSの普及により,誰もが画像や動画と併せて情報を発信・収集できるようになった.
逆に故意に情報を捏造して発信することによって,人々を誤った方向へ扇動するフェイクニュースも存在する.
虚偽の情報ながら,扇動ではなく皮肉や風刺を込めたジョークニュースも存在する.
ジョークニュースはフェイクニュースと同じく限りなく真実を模した形式をとるため,
昨今では同じくSNS上で拡散されやすく,同時に批判に晒されることもある.

% 本研究
本研究では,画像つきで発信された情報に対して,正しい情報か・フェイクニュースか・ジョークニュースかを判断するモデルを構築する.
このモデルを使い,従来から画像・テキスト複合のデータセットに対して3カテゴリでも優秀な分類が行えることを示すことを目指す.
それにより,SNSユーザの情報収集を支援するエージェントの開発につなげることが可能となる.
