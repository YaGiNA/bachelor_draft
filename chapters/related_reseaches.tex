\section{関連研究}\label{ch:related}
% フェイクニュースのテキストによる分類
正しいニュース・フェイクニュース・ジョークニュースの3カテゴリ分類を機械学習で行う研究がある\cite{DBLP:journals/corr/HorneA17}.
別の対象として,テキスト・画像を併せた情報を分類する機械学習モデルの検討も数多く行われている\cite{Wang:2018:EEA:3219819.3219903}.
画像・テキスト双方を扱うモデルでは,
実際に真実・フェイクとのカテゴリ分類において画像単独・テキスト単独の分類に比べて優秀な成績を収めていた
\cite{Wang:2018:EEA:3219819.3219903}.
しかしながら,あくまで``真実なのかそうでないのか''という2カテゴリで分類しているため,
``他者を欺くための情報なのか,皮肉・風刺を込めた情報なのか''という観点での分析がなされていない.
今回対象とする情報は,SNS上で投稿された画像つきで発信されたニュースである.
そのなかでも,正しいニュースを発信していたもの,フェイクニュースを発信していたもの,ジョークニュースを発信していたものが対象となる.
