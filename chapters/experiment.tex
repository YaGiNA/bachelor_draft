%
\section{評価実験}\label{ch:experiment}
%
\subsection{実験条件}
今実験ではTwitterデータセットを使用した\cite{boididou2015verifying}.
% 

画像つき文章投稿を3カテゴリに分類する提案手法の有効性を調べるために
文章のみで投稿を分類する手法(以降,Textと表記),もう1つは画像のみで投稿を分類する手法(以降,Imageと表記)を用意した.
いずれも上記提案モデルから文章・画像特徴抽出器を除外したモデルを使用した.
Textは入力データを提案モデルが使用したデータセットから画像を削除したものを使用し,
Imageは全投稿で使用された画像を対象とし,同じ画像に対して複数の文章投稿があった場合は1件として数えることにした.
上記の条件を踏まえ,提案手法・Text・Imageが扱う3カテゴリの投稿件数は以下の表\ref{table:posts}の通りである.

\begin{table}[h]
    \caption{提案手法と比較対象手法が扱うカテゴリ毎の投稿数}
    \label{table:posts}
    \centering
    \begin{tabular}{clll}
        \hline
        手法 & Real & Fake & Humor \\
        \hline \hline
        Text & 3021 & 4233 & 1509 \\
        Image & 172 & 157 & 82 \\
        提案手法 & 3021 & 4233 & 1509 \\
        \hline
    \end{tabular}
\end{table}

\subsection{実験結果}
3モデルに対して10分割交差検定を行った結果が以下の表\ref{table:result}の通りである.
評価指標では,Precision(精度), Recall(再現率), F値(左2値の調和平均)のマクロ平均を使用することにした.
% 
\begin{table}[h]
    \caption{各モデルの分類成果(マクロ平均)}
    \label{table:result}
    \centering
    \begin{tabular}{clll}
        \hline
        手法 & Precision & Recall & F値 \\
        \hline \hline
        Text & 0.3649 & 0.3677 & 0.3016 \\
        Image & 0.4942 & 0.5055 & 0.4667 \\
        提案手法 & 0.9268 & 0.9362 & 0.9286 \\
        \hline
    \end{tabular}
\end{table}

この結果を見ると,提案手法が他2手法と比べて非常に高い分類成果を挙げたことが読み取れた.
