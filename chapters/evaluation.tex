\section{評価}\label{ch:evaluate}

\subsection{考察}
今回の評価実験では,提案手法が3指標全てにおいて比較対象手法より優れた分類成績を収めた.
これにより,SNS上で画像つきの投稿を対象にした場合,正しいニュース・フェイクニュースの分類タスクのみならず,
ジョークニュースも含めた分類においても従来のマルチメディアモデルのアプローチが有効であることが示唆されたのではないかと考えられる.

\subsection{課題}
今回分類するにあたり,大きな課題となったのが文章投稿の単語埋め込みへの変換であった.
データセットがTwitterから収集されたものであったため,
ユーザ生成コンテンツに対応することが難しかった.
さらに,このモデルは英語のみを対象とした影響で,
データセット内他国語投稿に対して対応ができなかった.

また,このモデルに限らずフェイクニュース検出というタスクにおいては,Wangらの研究\cite{Wang:2018:EEA:3219819.3219903}によって問題点が指摘されていた.
訓練データが扱うイベントや出来事の特殊性の影響を受けることにより,
検証する時に訓練になかった別のイベントや出来事に対して正常な判断ができなくなる点であった.
%