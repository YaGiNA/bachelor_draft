%
\section{おわりに}\label{ch:conclusion}
%
\subsection{本論文のまとめ}
本研究では,SNS上で画像と文章を併せて発信された情報に対して,正しいニュース・フェイクニュース・ジョークニュースを判断するモデルを提案した.
実際に3カテゴリ分類を行った結果,文章・画像単体から分類した場合に比べて,全ての評価指標において非常に優秀な分類成績を挙げた.
これによりSNS上における画像つき投稿に対して,ジョークニュースを含めた3カテゴリ分類も有効であることが示された.
%
\subsection{今後の展望}
このモデルの発展として,いくつかの方法が考えられる.

データセットが扱う出来事やイベントによる特殊性の対策として,
Wangらの研究\cite{Wang:2018:EEA:3219819.3219903}では敵対的生成ネットワーク(GAN)を模倣する形をとることが挙げられていた.
真偽分類に加えて扱われたイベントも分類することによって,
フェイクニュースの普遍的な特徴を抽出するようなアプローチが行われていた.

提案手法を日本語投稿に対応させることを考えた場合,
残念ながら国内に今回使用したデータセットに近い規模をもつものがないため,
SNS上で日本語による画像つきの3カテゴリの投稿を収集する必要がある.
% 